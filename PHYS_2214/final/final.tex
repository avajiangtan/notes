\documentclass{hw}
\title{PHYS 2214 -- Final}
\author{community}

\usepackage{code}
\usepackage{pgf}
\usepackage{tikz}

\numberwithin{equation}{section}

%%%%%%%%%%%%%%%%%%%%%%%%%%%%%%%%%%%%%%%%%%%%%%%%%%%%%%%%%%%%%%%%%%%%%%%%%%%%%%%%
\begin{document}
\maketitle

\tableofcontents
\newpage{}

\section{Geometric Optics}

\section{Power and Momentum in EM Waves}
\subsection{Power and the Poynting Vector}
We have already explored the power of mechanical waves and found that it was
the product of the energy density and wave speed. We did not explore the
momentum of such waves but we can imagine that mechanical waves also have
momentum. This can be shown by running the simple experiment of setting a ball
in motion at the end of an oscillating string.

Similarly, in EM we can talk about energy and momentum. We first start with the
notion of the volumetric energy density of EM waves (Electric Field and Magnetic
Field Components):
\begin{equation}
\cbox{%
  u(t)  = \frac{1}{2}\epsilon_0 E(t)^2+\frac{1}{2\mu_0}B(t)^2
}
\end{equation}

$\epsilon_0$ and $\mu_0$ are properties of Electric Field and Magnetic fields
in a vacuum. More general constants are $\epsilon$ and $\mu$

Since we know that 
\begin{equation}
\cbox{%
  B(t) = \frac{E(t)}{c} = \sqrt{\epsilon_0 \mu_0} E(t)
}
\end{equation}
We can say that:
\begin{align}
  u(t) &= \frac{1}{2} \epsilon_0 E(t)^2 + \frac{1}{2 \mu_0} B(t)^2      \\
  u(t) &= \frac{1}{2} \epsilon_0 E(t)^2 +
          \frac{1}{2 \mu_0} \epsilon_0 \mu_0 E(t)^2                     \\
  u(t) &= \frac{1}{2} \epsilon_0 E(t)^2 + \frac{1}{2} \epsilon_0 E(t)^2 \\
  u(t) &= \cbox{\epsilon_0 E(t)^2}
\end{align}

Note that $E$ and $B$ are magnitudes of the Electric Field and Magnetic Field
respectively. Since these magnitudes are a function of position and time, the
energy density $u$ is also in general a function of position and time.

In EM it is often useful to talk about the \textit{erngy transfered per unit
time per unit cross-sectional area} or \textit{power per unit area}. The area
is perpendicular to the direction of the EM wave's propagation (velocity). We
now get that:

\begin{align}
  S &= \frac{\text{Power}}{{Area}}             \\
    &= \frac{dU}{dt} \cdot \frac{1}{A}         \\
    &= u(t) c                                  \\
    &= \epsilon_0 cE^2                         \\
    &= \epsilon_0 c E Bc                       \\
    &= \frac{\epsilon_0}{\epsilon_0 \mu_0} E B \\
    &= \cbox{\frac{\left|E\right| \left|B\right|}{\mu_0}} \label{eq:PoyntingMag}
\end{align}

Here we see that $S$ is a scalar quantity. If we define a vector quantity, we
get that:

\begin{equation}
\cbox{%
  \vec{S}(t) = \frac{1}{\mu_0} \vec{E}(t) \times \vec{B}(t)
}
\end{equation}

The vector quantity $\vec{S}$ represents the \textbf{Poynting Vector}. We can
know show Equation~\ref{eq:PoyntingMag} by noting that $\vec{E}$ and $\vec{B}$
are perpendicular. Also notice that the Poynting vector is a function of time.

The \textbf{Intensity} of an EM wave is defined as the average power per unit
area. In addition, by using the fact that $\vec{E}$ and $\vec{B}$ oscillate in
position and time, we can show that: 
\begin{equation}
\cbox{%
  Intensity = S_{av} = \frac{E_{max} B_{max}}{2\mu_0} = \frac{E_{max}^2}{2\mu_0
  c} = \frac{c\epsilon_0 E_{max}^2}{2}
}
\end{equation}
Here, the factor of $\frac{1}{2}$ comes from taking the average value of $E$
and $B$, both of which are sinusoidal.

\subsection{Momentum}
Electromagnetic waves have momentum, denoted $p$. Here, we consider the
momentum flow per unit area of an electromagnetic wave.
\begin{equation}\label{eq:radiation-pressure}
  \frac{\text{momentum flow}}{\text{area}} 
    = \frac{dp}{dt} \cdot \frac{1}{\text{area}}
\end{equation}

The units of Equation~\ref{eq:radiation-pressure} are $\frac{N}{m^2}$, or
pressure. Equation~\ref{eq:radiation-pressure} describes the \emph{radiation
pressure} of a wave: the pressure it exerts on a surface with which it
collides.

If the electromagnetic wave is completely absorbed by the surface it hits, then
\begin{equation}
  \cbox{\text{radiation pressure} = \frac{S}{c}}
\end{equation}
If the electromagnetic wave is reflected, then 
\begin{equation}
  \cbox{\text{radiation pressure} = \frac{2S}{c}}
\end{equation}

%%%%%%%%%%%%%%%%%%%%%%%%%%%%%%%%%%%%%%%%%%%%%%%%%%%%%%%%%%%%%%%%%%%%%%%%%%%%%%%%
\section{Interference}\label{sec:interference}
Earlier, we explored and analyzed electromagnetic waves through the lens of
geometric optics. In the following sections, we'll study electromagnetic waves
through the lens of \emph{physical optics}. Physical optics revolves around
viewing light as waves and observing the phenomena that arise through this
viewpoint. For example, we'll see interference and diffraction.

\subsection{Interference in Two or Three Dimensions}
Consider two sources of light, $S_1$ and $S_2$. The two sources produce
\emph{coherent} light: light waves that have the same frequency and
polarization and have a constant phase relationship. 

$S_1$ and $S_2$ are not at the same location. That means that if we observe the
light they produce at some point, the light from each source travels a
different distance. Denote the difference in their path lengths $\Delta r$. We
observe constructive interference when 
\begin{equation}\label{eq:constructive}
  \cbox{\Delta r = m \lambda, \quad m \in \mathbb{Z}}
\end{equation}

We observe destructive interference when
\begin{equation}\label{eq:destructive}
  \cbox{\Delta r = \group{m + \frac{1}{2}} \lambda, \quad m \in \mathbb{N}}
\end{equation}

We suggest you take a moment to internalize these equations. Multiple equations
for $n$-source interference and diffraction will be derived from this intuitive
pair of equations. Remembering where they come from will help keep everything
sorted out.

\subsection{Two-Source Interference of Light}
Thomas Young's two-slit experiment is an example of two-source interference.
Two coherent light sources at a distance $d$ are aimed at a surface a large
distance away. 

To find the location of constructive and destructive interference, we'll simply
use Equation~\ref{eq:constructive} and Equation~\ref{eq:destructive}. The path
difference at an angle $\theta$ is $d \sin(\theta)$. We'll substitute this
value for $\Delta r$ and get equations for constructive and destructive
interference.

\begin{gather}
  \cbox{d \sin(\theta) = m \lambda} \label{eq:interference-constructive} \\
  \cbox{d \sin(\theta) = \group{m + \frac{1}{2}} \lambda}
\end{gather}

The patterns formed on the screen are a succession of bright and dark bands, or
\emph{interference fringes}. Denote the vertical distance to the center of the
bright band $m$, $y_m$. Denote the distance from the two light sources to the
surface $R$. Using the small angle approximation, we can create a formula for
$y_m$.

\begin{equation}
  \cbox{y_m = R \frac{m \lambda}{d}}
\end{equation}

Notice that $\frac{m \lambda}{d} = \theta_m$.

%%%%%%%%%%%%%%%%%%%%%%%%%%%%%%%%%%%%%%%%%%%%%%%%%%%%%%%%%%%%%%%%%%%%%%%%%%%%%%%%
\section{Diffraction}
\subsection{What is diffraction?}
Consider a beam of light striking a single slit. Thanks to Huygens's principle,
we can think of the light emerging from the slit as many sources of light. In
\S~\ref{sec:interference}, we analyzed two-source interference. The analysis of
diffraction is is identical, except for the fact that there are more sources.

It begs the question, \emph{what's the difference between interference and
diffraction}? Nothing! Interference is the superposition of light waves.
Two-source interference is the interference of two light sources. Diffraction
is the interference of many light sources. The two phenomena are equivalent.  

\subsection{Diffraction Fringes}
Two-source interference produced interference fringes of equal spacing and
nearly equal intensity. Diffraction produces fringes too, but they're quite a
bit different. First off, the center band a big potato. It is much larger than
the other bands. Second off, the other bright bands are much dimmer than the
central potato. 

When we analyzed two-source interference, we found an equation for the location
of the fringes. Now, with such a big potato in the center, it doesn't make as
much sense to give the position of the bright bands. Instead, we'll find an
equation for the position of the dark bands.

Again, we'll use Equation~\ref{eq:constructive} and
Equation~\ref{eq:destructive}. Denote the width of the slit $a$\footnote{not to
be confused with $d$ from \S~\ref{sec:interference} which was the distance
between slits}. The equation for destructive interference is

\begin{equation}\label{eq:diffraction-destructive}
  \cbox{a \sin(\theta) = m \lambda, \quad m \in \mathbb{Z} - \set{0}}
\end{equation}

Notice a couple subtleties.
\begin{itemize}
  \item Equation~\ref{eq:diffraction-destructive} looks very similar to
    Equation~\ref{eq:interference-constructive}. However, the former equation
    is for dark bands and the latter is for bright bands. The reason they are
    different is explained in the book and is a bit hand wavy.
  \item Here $m \in \mathbb{Z} - \set{0} = \set{-1, 1, -2, 2, \ldots}$. $m$
    cannot be $0$ because there is no dark band in the middle of the potato.
\end{itemize}

%%%%%%%%%%%%%%%%%%%%%%%%%%%%%%%%%%%%%%%%%%%%%%%%%%%%%%%%%%%%%%%%%%%%%%%%%%%%%%%%
\section{Interference and Diffraction Intensities}
\subsection{Two-source interference}
Let $\phi$ be the phase shift between the two light sources. Let $E_p$ be the
electric field magnitude of the superimposed wave.

\begin{equation}
  \cbox{E_p = 2E\left| \cos \frac{\phi}{2} \right|}
\end{equation}

\begin{equation}
  \cbox{%
    I = 2 \epsilon_0 c E^2 \cos^2 \frac{\phi}{2} = I_0 \cos^2 \frac{\phi}{2}
  }
\end{equation}

\begin{equation}
  \cbox{%
    \phi = 2 \pi \frac{\Delta r}{\lambda} = 2 \pi \frac{d \sin(\theta)}{\lambda}
  }
\end{equation}

These interference fringes make a nice looking sinusoid.

\subsection{One-slit Diffraction}
Here $\beta$ plays the role of $\phi$. $E_0$ is the electric field component of
the incoming wave.
\begin{equation}
  \cbox{E_p = E_0 \frac{\sin(\beta / 2)}{\beta / 2}}
\end{equation}

\begin{equation}
  \cbox{I = I_0 \left[ \frac{\sin(\beta / 2)}{\beta / 2}^2 \right]}
\end{equation}

\begin{equation}
  \cbox{\beta = 2 \pi \frac{a \sin(\theta)}{\lambda}}
\end{equation}

%%%%%%%%%%%%%%%%%%%%%%%%%%%%%%%%%%%%%%%%%%%%%%%%%%%%%%%%%%%%%%%%%%%%%%%%%%%%%%%%
\end{document}
